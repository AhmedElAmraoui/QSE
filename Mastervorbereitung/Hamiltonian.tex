\documentclass[12pt]{article}
\usepackage[a4paper,margin=2.5cm]{geometry}
\usepackage{amsmath, amssymb}
\usepackage{physics}
\usepackage{bm}
\usepackage{graphicx}
\usepackage{hyperref}
\usepackage{braket}
\usepackage{enumitem}

\title{\textbf{Erklärung des QC$^4$H$_2$O-Hamiltonians}}
\author{}
\date{}

\begin{document}
	
	\maketitle
	
	\section*{Überblick}
	
	Dieser Text erläutert die Terme des Hamiltonoperators, der im Projekt \textit{QC$^4$H$_2$O – Simulating Water on a Quantum Computer} verwendet wird. Dabei wird die gekoppelte Beschreibung von elektronischen und protonischen Freiheitsgraden auf einem Quantencomputer angestrebt.
	
	\section*{Hamiltonian}
	
	Der vollständige Hamiltonoperator hat folgende Form:
	
	\begin{align}
		\mathcal{H} = & \sum_p v^{pO} \hat{\sigma}_p^z
		+ \frac{1}{2} \sum_{p \neq q} v^{pq} \hat{\sigma}_p^z \hat{\sigma}_q^z
		+ \sum_{i,j} \left[t_{ij} + v_{ij}^O \right] a_i^\dagger a_j
		+ \sum_{i,j,p} v_{ij}^p a_i^\dagger a_j \hat{\sigma}_p^z \nonumber \\
		& + \frac{1}{2} \sum_{i,j,k,l} v_{ijkl} a_i^\dagger a_j^\dagger a_l a_k
		- \mu_{\mathrm{H}^{+}} \sum_p \hat{\sigma}_p^z
		- \mu_{\mathrm{e}^{-}} \sum_i a_i^\dagger a_i
	\end{align}
	
	\section*{Erklärung der Terme}
	
	\begin{enumerate}[label=\textbf{(\arabic*)}, leftmargin=1.5cm]
		\item \textbf{Proton–Sauerstoff-Wechselwirkung:}
		\[
		\sum_p v^{pO} \hat{\sigma}_p^z
		\]
		Dieser Term beschreibt die Coulomb-Anziehung zwischen einem Proton $p$ und dem Sauerstoffkern. Die Besetzungsvariable $\hat{\sigma}_p^z$ (Pauli-Z-Operator) nimmt die Werte $\pm1$ bzw. $0$ oder $1$ für besetzt/unbesetzt an.
		
		\item \textbf{Proton–Proton-Wechselwirkung:}
		\[
		\frac{1}{2} \sum_{p \neq q} v^{pq} \hat{\sigma}_p^z \hat{\sigma}_q^z
		\]
		Repräsentiert die elektrostatische Abstoßung zwischen zwei verschiedenen Protonen.
		
		\item \textbf{Elektronenkinetik und Elektron–Sauerstoff-Wechselwirkung:}
		\[
		\sum_{i,j} \left[t_{ij} + v_{ij}^O \right] a_i^\dagger a_j
		\]
		\begin{itemize}
			\item $t_{ij}$: Kinetische Energie der Elektronen (kinetischer Operator).
			\item $v_{ij}^O$: Coulomb-Anziehung zwischen Elektronen und dem fixen Sauerstoffkern.
		\end{itemize}
		
		\item \textbf{Elektron–Proton-Wechselwirkung:}
		\[
		\sum_{i,j,p} v_{ij}^p a_i^\dagger a_j \hat{\sigma}_p^z
		\]
		Dieser Term beschreibt die Wechselwirkung zwischen Elektronen und den einzelnen Protonen, abhängig von deren Besetzung.
		
		\item \textbf{Elektron–Elektron-Wechselwirkung:}
		\[
		\frac{1}{2} \sum_{i,j,k,l} v_{ijkl} a_i^\dagger a_j^\dagger a_l a_k
		\]
		Repräsentiert die Coulomb-Wechselwirkungen zwischen Paaren von Elektronen (Zweikörperterm).
		
		\item \textbf{Chemisches Potential für Protonen:}
		\[
		- \mu_{\mathrm{H}^{+}} \sum_p \hat{\sigma}_p^z
		\]
		Ermöglicht Kontrolle über die Anzahl der Protonen im System (z.\,B. Protonentransfer).
		
		\item \textbf{Chemisches Potential für Elektronen:}
		\[
		- \mu_{\mathrm{e}^{-}} \sum_i a_i^\dagger a_i
		\]
		Dient zur Kontrolle der Elektronenzahl im System (z.\,B. für Ladungsneutralität oder Ionen).
	\end{enumerate}
	
	\section*{Zusätzliche Hinweise}
	
	\begin{itemize}
		\item Die Operatoren $a_i^\dagger, a_i$ sind Fermion-Erzeugungs- und -Vernichtungsoperatoren für Spinorbitale.
		\item Die $\hat{\sigma}_p^z$ sind Pauli-Z-Operatoren, die als Besetzungsvariablen für Protonen verwendet werden.
		\item Die Indizes $i, j, k, l$ laufen über elektronische Spinorbitale; $p, q$ über Protonensites.
		\item Der Hamiltonian ist in Born-Oppenheimer-Näherung formuliert – der Sauerstoffkern ist fixiert.
	\end{itemize}
	
\end{document}
